% Start preamble
 
 \documentclass[12pt]{article}
  
  \usepackage[margin=1in]{geometry} 
  \usepackage{amsmath,amsthm,amssymb}
  \usepackage{siunitx}
  \usepackage[table,xcdraw]{xcolor} % needed for the table colors
  %\usepackage[options]{natbib} % for modifications to the citations

  % suppress paragraph indent
  \setlength\parindent{0pt}

  % give some space between paragraphs
  \setlength{\parskip}{3mm plus1mm minus1mm}

   \newcommand\SNR{\mathit{SNR}} %make it so that SNR can be a single symbol 
   \newcommand{\N}{\mathbb{N}}
   \newcommand{\Z}{\mathbb{Z}}
    \newenvironment{exercise}[2][Exercise]{\begin{trivlist}
    \item[\hskip \labelsep {\bfseries #1}\hskip \labelsep {\bfseries #2.}]}{\end{trivlist}}
     
     % end of the preambles
     \begin{document}
      
      % --------------------------------------------------------------
      %                         Start here
      % --------------------------------------------------------------
       
      
        
	\title{Final Exam} 
	\author{Jonathan King\\ %
	SASENG1170 - Introduction to Radar Systems} %
	 
	 \maketitle

   \section{Answers}
   \begin{table}[h]
     \centering
     \caption{My caption}
     \label{my-label}
     \begin{tabular}{|c|c|c|c|c|c|}
      \hline
      \rowcolor[HTML]{9B9B9B} 
      Question & Answer & Question & Answer & Question & Answer \\ \hline
      1        & B      & 6        & D      & 11       & D      \\ \hline
      2        & A      & 7        & A      & 12       & C      \\ \hline
      3        & B      & 8        & D      & 13       & B      \\ \hline
      4        & D      & 9        & C      & 14       & D      \\ \hline
      5        & C      & 10       & B      & 15       & C      \\ \hline
     \end{tabular}
    \end{table}

    \section{Show Your Work...}

      % ---------------------------------------------------------------------------------------- 
      % done
      \begin{exercise}{1}
      What wavelength corresponds to a frequency of 17Ghz?
      
      To solve, use the following relationship between wavelength, speed and frequency:
      \begin{align*}
      \lambda & = \frac{c}{f} \\
      & = \frac{\SI{299792458}{\meter\per\second}}{\SI{17}{\GHz}}\\
      & = \SI{1.763}{\cm}
      \end{align*}
      
      \end{exercise}
      
      % ---------------------------------------------------------------------------------------- 
      % FIX ME
      \begin{exercise}{2}
      What would the gain of a circular antenna with a diameter of 30cm and a wavelength of 1cm be? (Assume 100\% efficiency)
      
      The relationship between antenna gain, area and wavelength is found in Equation 2.7 from \cite[p.~63]{POMR}:
      \begin{align*}
      G_{R} & = 4 \pi \frac{ A_{e} }{\lambda^{2}} \\
      & = 4 \pi^{2} \frac{ R^{2}}{\lambda^{2}}\\
      & = 4 \pi^{2} \frac{ (\SI{0.15}{\meter})^{2}}{(\SI{.01}{\meter})^{2}}\\
      & = 4 \pi^{2} \frac{ (\SI{0.0225}{\meter\squared}) }{ (\SI{.0001}{\meter\squared}) }\\
      & = \SI{8882.644}{}\\
      \end{align*}      
      \end{exercise}
      
      % ---------------------------------------------------------------------------------------- 
      % done
      \begin{exercise}{3}
      If a radar beam scans past a target at 80 deg/sec.  If the beamwidth is 3 degrees, what is the dwell time?

      The relationship between scan rate, beam width and dwell time is found in Equation 12.2 from \cite[p.~88]{POMR}:
      \begin{align*}
      T_{ad} & = \frac{\theta_{3 dB}}{\omega}\\
      & = \frac{\SI{3}{\degree}}{\SI{80}{\degree\per\second}}\\
      & = \SI{0.0375}{\second}
      \end{align*}      
      \end{exercise}

      % ---------------------------------------------------------------------------------------- 
      % done
      \begin{exercise}{4}
      Which of the following is not a factor in the radar range equation?

      The radar equation is given by equation 2.17 from \cite[p.~68]{POMR}:
      \begin{align*}
      \frac{P_{r}}{P_{n}} & = \frac{ P_{t} G_{t} G_{r} \lambda^{2} \sigma n_{p} }{ (4\pi)^{3} R^{4} k T_{0} B F L_{S} }
      \end{align*}
      Clearly the wavelength ($\lambda$), the range (R) and the losses (F and $L_{S}$) are part of the radar range equation. The probability  of detection is not a factor in the radar range equation.
      \end{exercise}

      % ---------------------------------------------------------------------------------------- 
      % could use a cite but mostly done
      \begin{exercise}{5}
      In problems 14 and 15, what would be the minimum target range with the 10:1 pulse compression?

      The radar in question uses a 10:1 pulse compression code modulation on a \SI{10}{\us} pulse. Since the radar cannot receive during transmission, the pulse width sets determines the minimum target range. Even though it employs a 10:1 pulse ratio, the pulses themselves are still \SI{10}{\us} wide. The pulse must be done transmitting by the time the wavefront of the pulse has hit the target and reflected back to the antenna, therefore $\frac{ 2 R_{min} }{c} = \tau$. Rearranging for $R_{min}$ we have
      \begin{align*}
      R_{min} & = \frac{\tau * c}{2}\\
      & = \frac{ (\SI{10}{\us}) (\SI{299792458}{\meter\per\second}) }{ 2 }\\
      & = \SI{1498.962}{\meter}\\
      \end{align*}
      \end{exercise}

      % ---------------------------------------------------------------------------------------- 
      % done
      \begin{exercise}{6}
      A circular antenna has a diameter of 30 cm. What would it's beamwidth be for a wavelength of 1 cm 

      A seen in \cite[p.~113]{IAR}, A uniformly illuminated circular aperture of diameter \textit{d} has the following 3 dB beamwidth
      \begin{align*}
      \Theta_{3 dB} & = 1.02 * \frac{\lambda}{d}\\
      & = 1.02 * \frac{(\SI{1}{\cm})}{(\SI{30}{\cm})}\\
      & = \SI{0.034}{\radian}\\
      & = \frac{\SI{180}{\degree}}{\pi}\SI{0.034}{\radian}\\
      & = \SI{1.948}{\degree}
      \end{align*}
      \end{exercise}

      % ---------------------------------------------------------------------------------------- 
      % done
      \begin{exercise}{7}
      A radar has satisfactory detection performance at 50 kM.  How much could the range be increased if the transmitter power were to be doubled?

      Using the single pulse version of the radar equation as given by equation 2.17 from \cite[p.~68]{POMR}, we can see the relationship between the two detection ranges as the power is doubled.

      \begin{align*}
      \frac{P_{t1}G^{2}\lambda^{2}\sigma n_{p}}{(4\pi)^{3}(R_{1})^{4} k T_{0} B F L_{S}} & = \frac{(P_{t2})G^{2}\lambda^{2}\sigma n_{p}}{(4\pi)^{3}(R_{2})^{4} k T_{0} B F L_{S}}\\
      \frac{P_{t1}}{(R_{1})^{4}} & = \frac{P_{t2}}{(R_{2})^{4}}\\
      R_{2} & = \left[\frac{P_{t2}(R_{1})^{4})}{P_{t1}}\right]^{\frac{1}{4}} \text{(Remember that $P_{t2} = 2 P_{t1}$)}\\
      & = \left[2(R_{1})^{4}\right]^{\frac{1}{4}}\\
      & = \left[2\right]^{\frac{1}{4}}\SI{50}{\km}\\
      & = \SI{59.460}{\km}
      \end{align*}
      Therefore, doubling the power increases the range by about 20\%.
      \end{exercise}
      
      % ---------------------------------------------------------------------------------------- 
      % done
      \begin{exercise}{8}
      What SNR is required for $P_{D}$ = 50\% at a PFA of $10^{-6}$

      For a Swerling 1 target in white noise with detection based on only a single sample, we can use equation 3.20 \cite[p.~108]{POMP}:
%
      \begin{align*}
      P_{D} & = (P_{FA}) ^ {\frac {1} { (1+\SNR)} }\\
      \log{(P_{D})} & = \frac {1} { (1+\SNR)} \log{(P_{FA})}\\
      \SNR & = \frac{ \log{( P_{FA} )} } { \log{(P_{D})} } - 1\\
      & = \frac{ \log{( 10^{-6} )} } { \log{(.50)} } - 1\\
      & = \frac{-6}{-0.301} - 1\\
      & = 18.932\\
      \end{align*}
      Since this isn't one of the available answers, I conclude that the conditions assumed by the equation as detailed in the text do not hold for this question. 

      Somewhat confusingly, Table 3.2 \cite[103]{POMR} includes solutions for the case where the $P_{D}$ = 50\% and the PFA of $10^{-6}$ for Swerling types SW0 through SW4. In these five cases, the answers range from 11.1 to 12.8 depending upon the type of Swerling model used. While one of the multiple choice answers agrees with this table, I cannot explain why it does not agree with Equation 3.2 given in the previous analysis. 

      \end{exercise}

      % ---------------------------------------------------------------------------------------- 
      % done
      \begin{exercise}{9}
      What is the best method for combining multiple pulse echoes from the standpoint of detection performance

      Four choices: M of N, Video Integration, Coherent Integration, Pick Largest Echo

      Pick largest echo is clearly the least effective as it does not rely on any integration to reduce the noise threshold. Picking the largest echo might indeed pick out a target but, depending upon the threshold could also trigger on a noise return.

      M of N \cite[p.~109]{POMR} relies on determining a detection at a certain range cell in M out of N pulses. This method significantly increases the $P_{D}$ while decreasing the $P_{FA}$ at the cost of taking additional time due to subsequent dwells. While this technique does not increase the SNR of the radar, it does improve detection sensitivity of the system by reducing the SNR required for detection. While this is an admirable feat, it is still fundamentally combining the detection results from several samples and thereby ignoring the pulse to pulse phase information inherent in the signal.

      \begin{quote}
      Integration is the process of combing multiple samples of a signal, each contaminated by noise or other interference, to ``average down the noise'' and obtain a single combined signal-plus-noise sample that has a higher SNR than the individual samples. Integration can be either coherent, meaning that the singal phase information is used, or noncoherent, meaning that only the magnitude of the signal is processed. \cite[p.~536]{POMR}
      \end{quote}
 
      Non-coherent or Video integration sums the amplitude at the each range bin across multiple samples. This result is then threshold tested to determine if a target exists. As per the text, ``Noncoherent integration is less efficient than coherent integration... because in discarding the data phase, noncoherent integration does not take advantage of all of the information in the data \cite[p.~538]{POMR}''  

      Coherent integration or coherent video integration relies on the fact that since the phase of each transmission is continuous and coherent, the resultant echoes can be split into the in-phase (I) and quadrature (Q) parts and summed. When the velocity of a target is known, the phase of each echo can be corrected before the vector sum. In the non-coherent video integration, a target's amplitude my have summed to zero, but in the non-coherent case, the IQ data can be corrected to give the full return of the object. While coherent integration of N data samples can theoretically increases the SNR by a factor of N \cite[p.~536]{POMR}, realistically this isn't usually achievable since the ``phase relationship between the data in two different CPIs is generally not known \cite[p.~550]{POMR}''

      While the definition of ``detection performance'' seems a bit nebulous, I would say that coherent integration is the best method for combining multiple pulse echoes from the standpoint of detection performance though in a real radar, a combination of these techniques combined with intra-pulse modulation is likely the most optimal choice.
      \end{exercise}
      
      % ---------------------------------------------------------------------------------------- 
      % done 
      \begin{exercise}{10}
      A target is moving away from a radar at \SI{50}{\meter\per\second}. What would the Doppler shift be at $\lambda$ = \SI{.03}{\meter}

      Using the following equation from \cite[p.~261]{POMR}, we see that
      \begin{align*}
      f_{D} & = -2\frac{\dot{R}}{\lambda} & \text{(Where $\dot{R}$ is the range rate)}\\
      & = -2\frac{ (\SI{50}{\meter\per\second}) }{ (\SI{.03}{\meter}) }\\
      & = \SI{3.333}{\kHz}
      \end{align*}
      \end{exercise}
      
      % ---------------------------------------------------------------------------------------- 
      % done
      \begin{exercise}{11}
      What is the limit on PRF for unambiguous range measurement at \SI{50}{\km}?

      Using equation 1.14 \cite[p.~22]{POMR}, we see that the unambiguous range is related to the PRF by
      \begin{align*}
      PRF & = \frac{c}{2 R_{unambiguous}}\\
      & = \frac{\SI{299792.458}{\km\per\second}}{\SI{100}{\km}}\\
      & = \SI{2997.924}{\Hz}
      \end{align*}
      \end{exercise}
      
      % ---------------------------------------------------------------------------------------- 
      % done
      \begin{exercise}{12}
      What would the radar cross section of an aluminum sphere of 30 cm diameter be at 2 cm wavelength?
      
      Figure 7-2 \cite[p.~248]{POMR} gives a plot of the Mies series solution of normalized RCS of a perfectly conducting sphere as a function of radius a and wavelength $\lambda$
      
      The vertical axis is $\sigma/\pi a^{2}$ and the horizontal is $2\pi a / \lambda$. Our first task is calculate the value of the X axis:
      \begin{align*}
      X & = 2\pi\frac{ a}{\lambda}\\
      & = 2\pi\frac{(\SI{30}{\cm})}{(\SI{2}{\cm})}\\
      & = 94.248\\
      \end{align*}
      Now we use this to determine Y. Careful analysis of the plot shows that it stops abruptly at $X = 10$, far short of our needs. 
      \begin{align*}
      Y &= f(X) \\
      & = f(94.248)\\
      & = ?
      \end{align*}

      Luckily, as the value of X increases, the value of Y approaches 1. This makes since when $\lambda \ll$ the diameter of the sphere, a perfectly conducting sphere approximates an isotropic scatterer and the value of Y approaches 1.
      \begin{align*}
      %Y &= \lim_{d_{\lambda}\to\infty \ll d} f(X)\\
      Y &= \lim_{\frac{d}{\lambda}\to\infty} f(X) \simeq 1\\
      \end{align*}
      
      Given this relationship, we can use the scale of the Y axis to solve for $\sigma$:
      \begin{align*}
      \frac{\sigma}{\pi a^{2}} & = 1 \\
      \sigma & = \pi a^{2}\\
      & = \pi (\SI{0.30}{\meter})^{2}\\
      & = \SI{0.283}{\meter\squared}
      \end{align*}
      \end{exercise}
      
      % ---------------------------------------------------------------------------------------- 
      % done
      \begin{exercise}{13}
      In problem 3, what would be the minimum Doppler filter width?

      The minimum achievable bandwidth a Doppler filter is given by \cite[p.~299]{IAR} as:
      \begin{align*}
      BW_{3 dB} & = \frac{1}{t_{int}}
      \end{align*}

      Recall from problem 3 that the dwell time for a \SI{80}{\degree\per\second} scan with a \SI{3}{\degree} beamwidth is \SI{0.0375}{\second}. Since the filter integration time can be at most equal to the dwell time, the minimum achievable filter bandwidth is
      \begin{align*}
      BW_{3 dB} & = \frac{1}{ \SI{0.0375}{\second} }\\
      &= \SI{26.667}{\Hz}
      \end{align*}
      \end{exercise}
      
      % ---------------------------------------------------------------------------------------- 
      % Good but not great
      \begin{exercise}{14}
      A radar achieves a SNR=10 dB on a particular using a 10 usec unmodulated pulse.  What would the SNR be if we implemented a 10:1 pulse compression code modulation on this pulse?  

      Keeping in mind that this is a intra-pulse modulation and not a simple matter of integrating a number of pulses, we must focus on determining the gain caused by the pulse compression. Equation 2.28 from \cite[p.~74]{POMR} and 20.48 \cite[p.~788]{POMR} show this relationship as
      \begin{align*}
      \SNR_{pc} = \SNR_{u} \tau \beta
      \end{align*}
      where $\tau$ is the pulse length and $\beta$ is the pulse modulation bandwidth, the product being referred to as the \emph{time-bandwidth product}. Simplistically, the bandwidth is usually the reciprocal of $\tau$ which is not very illustrative as they would both cancel out. We also cannot rewrite the time-bandwidth product as $\tau / \tau_{comp}$. Rewriting equation 2.28 we find:
      \begin{align*}
      \SNR_{pc} & = \SNR_{u} \frac{\tau}{\tau_{comp}}  & \text{(Linear)} \\
      &= \SNR_{u} + 10 \log { (\frac{\tau}{\tau_{comp}}) } & \text{(Decibel)}\\
      &= \SNR_{u} + 10 \log { (10) }\\
      &= \SNR_{u} + 10\\
      &= \SI{20}{\dB}\\
      \end{align*}  
      We know this because 20dB is not one of the available answers!  What is useful is that the text indicates that, ``The net result is a gain i the SNR at the output of the matched filter\cite[p.~788]{POMR}''. Since there is only one answer greater than 10dB, I assume that the answer must be 13dB!  Working backwards, if the pulse bandwith product were known to be 2, we can show that
      \begin{align*}
      \SNR_{pc} & = \SNR_{u} (\tau \beta) & \text{(Linear)}\\
      &= \SNR_{u} + 10 \log {\tau \beta } & \text{(Decibel)}\\
      &= \SI{10}{\dB} + 10 \log { (2) }\\
      &= \SI{10}{\dB} + \SI{3.010}{\dB}\\
      &= \SI{13.010}{\dB}\\
      \end{align*}

      \end{exercise}
      
      % ---------------------------------------------------------------------------------------- 
      % done
      \begin{exercise}{15}
      In the problem 14, what would the range resolution be in meters with the 10:1 pulse compression?      

      The radar resolution without pulse compression is given by equation 1.19 from \cite[p.~29]{POMR}:
      \begin{align*}
      R_{resolution} & = \frac{PulseLength}{2}\\
      & = \frac{ \tau c }{2}\\
      & = \frac{ (\SI{10}{\us}) (\SI{299792458}{\meter\per\second}) }{2}\\
      & = \SI{1498.962}{\meter}\\
      \end{align*}
      With 10:1 pulse compression, the range resolution is simply multiplied by 10:
      \begin{align*}
      R_{resolution} & = \frac{PulseLength}{2 N_{p}}\\
      & = \frac{  \tau c }{2 N_{p}}\\
      & = \frac{ (\SI{10}{\us}) (\SI{299792458}{\meter\per\second}) }{20}\\
      & = \SI{149.896}{\meter}\\
      \end{align*}
      \end{exercise}

      % -----------------------------------------------------------------------------------------
      \begin{thebibliography}{9}

      \bibitem{POMR}
        Mark A. Richards,
        \emph{Principles of Modern Radar Volume 1: Basic Principles},
        SciTech Publishing Raleigh, NC
        1st Edition,
        2015.

      \bibitem{IAR}
        George W. Stimson,
        \emph{Introduction to Airborne Radar},
        SciTech Publishing Edison, NJ
        3rd Edition,
        2014.

      \end{thebibliography}

      % --------------------------------------------------------------
      %     You don't have to mess with anything below this line.
      % --------------------------------------------------------------
       
       \end{document}

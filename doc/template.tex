% --------------------------------------------------------------
% This is all preamble stuff that you don't have to worry about.
% Head down to where it says "Start here"
% --------------------------------------------------------------
 
 \documentclass[12pt]{article}
  
  \usepackage[margin=1in]{geometry} 
  \usepackage{amsmath,amsthm,amssymb}
  \usepackage{siunitx}

   \newcommand{\N}{\mathbb{N}}
   \newcommand{\Z}{\mathbb{Z}}
    
    \newenvironment{theorem}[2][Theorem]{\begin{trivlist}
    \item[\hskip \labelsep {\bfseries #1}\hskip \labelsep {\bfseries #2.}]}{\end{trivlist}}
    \newenvironment{lemma}[2][Lemma]{\begin{trivlist}
    \item[\hskip \labelsep {\bfseries #1}\hskip \labelsep {\bfseries #2.}]}{\end{trivlist}}
    \newenvironment{exercise}[2][Exercise]{\begin{trivlist}
    \item[\hskip \labelsep {\bfseries #1}\hskip \labelsep {\bfseries #2.}]}{\end{trivlist}}
    \newenvironment{reflection}[2][Reflection]{\begin{trivlist}
    \item[\hskip \labelsep {\bfseries #1}\hskip \labelsep {\bfseries #2.}]}{\end{trivlist}}
    \newenvironment{proposition}[2][Proposition]{\begin{trivlist}
    \item[\hskip \labelsep {\bfseries #1}\hskip \labelsep {\bfseries #2.}]}{\end{trivlist}}
    \newenvironment{corollary}[2][Corollary]{\begin{trivlist}
    \item[\hskip \labelsep {\bfseries #1}\hskip \labelsep {\bfseries #2.}]}{\end{trivlist}}
     
     \begin{document}
      
      % --------------------------------------------------------------
      %                         Start here
      % --------------------------------------------------------------
       
       %\renewcommand{\qedsymbol}{\filledbox}
        
	\title{Homework Template}%replace X with the appropriate number
	\author{Professor Sarah\\ %replace with your name
	MATH 301 - Introduction to Proof and Abstract Reasoning} %if necessary, replace with your course title
	 
	 \maketitle
	  
      % ---------------------------------------------------------------------------------------- 
      \begin{exercise}{1}
      What wavelength corresponds to a frequency of 17Ghz?
      
      \begin{align*}
      \lambda & = \frac{c}{f} & (\text{Relationship between wavelength, speed and frequency})\\
      & = \frac{\SI{299792458}{\meter\per\second}}{\SI{17}{\GHz}}\\
      & = \SI{1.763485047}{\cm}
      \end{align*}
      
      \end{exercise}
      
      % ---------------------------------------------------------------------------------------- 
      \begin{exercise}{2}
      What would the gain of a circular antenna with a diameter of 30cm and a wavelength of 1cm be? (Assume 100\% efficiency)
      
      \begin{align*}
      G_{R} & = 4 \pi \frac{ A_{e} }{\lambda^{2}} & (\text{Relationship between antenna gain, area and wavelength})\\
      & = 4 \pi^{2} \frac{ R^{2}}{\lambda^{2}}\\
      & = 4 \pi^{2} \frac{ (\SI{0.3}{\meter})^{2}}{(\SI{.01}{\meter})^{2}}\\
      & = 4 \pi^{2} \frac{\SI{0.09}{\meter\squared}}{\SI{.0001}{\meter\squared}}\\
      & = \SI{35530.576}{}\\
      \end{align*}      
            
      \end{exercise}
      
      % ---------------------------------------------------------------------------------------- 
      \begin{exercise}{3}
      If a radar beam scans past a target at 80 deg/sec.  If the beamwidth is 3 deg, what is the dwell time?
      \begin{align*}
      Time & = \frac{(Angular Distance)}{(Angular Velocity)}\\
      & = \frac{\SI{3}{\degree}}{\SI{80}{\degree\per\second}}\\
      & = \SI{0.0375}{\second}
      \end{align*}      
      \end{exercise}

      % ---------------------------------------------------------------------------------------- 
      \begin{exercise}{4}
      Which of the following is not a factor in the radar range equation?

      The radar equation is
      \begin{align*}
      \frac{P_{r}}{P_{n}} & = \frac{P_{t}G^{2}\lambda^{2}\sigma}{(4\pi)^{3}R^{4} k T_{0} B F}
      \end{align*}
      Clearly the wavelength $\lambda$, the range R and the losses F are part of the radar range eqation.  The probabilty of detection is not a factor in the radar range quation.
      \end{exercise}

      % ---------------------------------------------------------------------------------------- 
      % FIXME
      \begin{exercise}{5}
      In problems 14 and 15, what would be the minimum target range with the 10:1 pulse compression?
      \begin{align*}
      X & = Y\\
      & = Z
      \end{align*}
      \end{exercise}

      % ---------------------------------------------------------------------------------------- 
      \begin{exercise}{6}
      A circular antenna has a diameter of 30 cm. What would it's beamwidth be for a wavelength of 1 cm 

      A uniformly illuminated circular aperture of diameter \textit{d} has the following 3 dB beamwidth
      \begin{align*}
      \Theta_{3 dB} & = 1.02 * \frac{\lambda}{d}\\
      & = 1.02 * \frac{(\SI{1}{\cm})}{(\SI{30}{\cm})}\\
      & = \SI{0.034}{\radian}\\
      & = \frac{\SI{180}{\degree}}{\pi}\SI{0.034}{\radian}\\
      & = \SI{1.948}{\degree}
      \end{align*}
      \end{exercise}

      % ---------------------------------------------------------------------------------------- 
      \begin{exercise}{7}
      A radar has satisfactory detection performance at 50 kM.  How much could the range be increased if the transmitter power were to be doubled?
      \begin{align*}
      \frac{P_{t1}G^{2}\lambda^{2}\sigma}{(4\pi)^{3}(R_{1})^{4} k T_{0} B F} & = \frac{(P_{t2})G^{2}\lambda^{2}\sigma}{(4\pi)^{3}(R_{2})^{4} k T_{0} B F}\\
      \frac{P_{t1}}{(R_{1})^{4}} & = \frac{P_{t2}}{(R_{2})^{4}}\\
      R_{2} & = \left[\frac{P_{t2}(R_{1})^{4})}{P_{t1}}\right]^{\frac{1}{4}}\\
      R_{2} & = \left[2(R_{1})^{4}\right]^{\frac{1}{4}} & (\text{P$_{t2}$ is twice P$_{t1}$})\\
%      R_{2} & = \left[2^{\frac{1}{4}}\right]R_{1}\\
      R_{2} & = \left[2\right]^{\frac{1}{4}}\SI{50}{\km}\\
      R_{2} & = \SI{59.460}{\km}
      \end{align*}
      \end{exercise}
      
      % ---------------------------------------------------------------------------------------- 
      \begin{exercise}{8}
      What SNR is required for $P_{D}$ = 50\% at a PFA of $10^{-6}$

      When $P_{D}$ = 50\%, need to look up a swerling curve
      According to table 3.2 of book on page 103, the SNR for the case when the $P_{D}$ = 50\% and the PFA of $10^{-6}$ is 11.1 to 12.8 depending upon the type of Swerling model used.
      \end{exercise}

      % ---------------------------------------------------------------------------------------- 
      % FIXME
      \begin{exercise}{9}
      What is the best method for combining multiple pulse echos from the standpoint of detection performance
      \begin{align*}
      X & = Y\\
      & = Z
      \end{align*}
      \end{exercise}
      
      % ---------------------------------------------------------------------------------------- 
      \begin{exercise}{10}
      A target is moving away from a radar at \SI{50}{\meter\per\second}.  What would the doppler shift be at $\lambda$ = \SI{.03}{\meter}
      \begin{align*}
      f_{D} & = -2\frac{\dot{R}}{\lambda} & (\text{Where $\dot{R}$ is the range rate})\\
      & = -2\frac{\SI{50}{\meter\per\second}}{\SI{.03}{\meter}}\\
      & = \SI{3.333}{\kHz}
      \end{align*}
      \end{exercise}
      
      % ---------------------------------------------------------------------------------------- 
      \begin{exercise}{11}
      What is the limit on PRF for unambiguous range meas at \SI{50}{\km}?

      The unambiguous range is related to the PRF by
      \begin{align*}
      PRF & = \frac{c}{2 R_{unambiguous}}\\
      & = \frac{\SI{299792.458}{\km\per\second}}{\SI{100}{\km}}\\
      & = \SI{2997.924}{\Hz}
      \end{align*}
      \end{exercise}
      
      % ---------------------------------------------------------------------------------------- 
      % FIXME
      \begin{exercise}{12}
      What would the radar cross section of an aluminum sphere of 30 cm diameter be at 2 cm wavelength?
      \begin{align*}
      X & = Y\\
      & = Z
      \end{align*}
      \end{exercise}
      
      % ---------------------------------------------------------------------------------------- 
      % FIXME
      \begin{exercise}{13}
      In problem 3, what would be the minimum doppler filter width
      \begin{align*}
      X & = Y\\
      & = Z
      \end{align*}
      \end{exercise}
      
      % ---------------------------------------------------------------------------------------- 
      % FIXME
      \begin{exercise}{14}
      A radar achieves a SNR=10 dB on a particular using a 10 usec unmodulated pulse.  What would the SNR be if we implemented a 10:1 pulse compression code modulation on this pulse?  
      \begin{align*}
      X & = Y\\
      & = Z
      \end{align*}
      \end{exercise}
      
      % ---------------------------------------------------------------------------------------- 
      % FIXME
      \begin{exercise}{15}
      In the problem 14, what would the range resolution be in meters with the 10:1 pulse compression      
      \begin{align*}
      X & = Y\\
      & = Z
      \end{align*}
      \end{exercise}




      
	      
	      % --------------------------------------------------------------
	      %     You don't have to mess with anything below this line.
	      % --------------------------------------------------------------
	       
	       \end{document}
